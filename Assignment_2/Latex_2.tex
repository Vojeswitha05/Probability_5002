\documentclass[journal,12pt,twocolumn]{IEEEtran}

\usepackage{setspace}
\usepackage{gensymb}
\singlespacing
\usepackage[cmex10]{amsmath}

\usepackage{amsthm}

\usepackage{mathrsfs}
\usepackage{txfonts}
\usepackage{stfloats}
\usepackage{bm}
\usepackage{cite}
\usepackage{cases}
\usepackage{subfig}

\usepackage{longtable}
\usepackage{multirow}

\usepackage{enumitem}
\usepackage{mathtools}
\usepackage{steinmetz}
\usepackage{tikz}
\usepackage{circuitikz}
\usepackage{verbatim}
\usepackage{tfrupee}
\usepackage[breaklinks=true]{hyperref}
\usepackage{graphicx}
\usepackage{tkz-euclide}

\usetikzlibrary{calc,math}
\usepackage{listings}
    \usepackage{color}                                            %%
    \usepackage{array}                                            %%
    \usepackage{longtable}                                        %%
    \usepackage{calc}                                             %%
    \usepackage{multirow}                                         %%
    \usepackage{hhline}                                           %%
    \usepackage{ifthen}                                           %%
    \usepackage{lscape}     
\usepackage{multicol}
\usepackage{chngcntr}

\DeclareMathOperator*{\Res}{Res}

\renewcommand\thesection{\arabic{section}}
\renewcommand\thesubsection{\thesection.\arabic{subsection}}
\renewcommand\thesubsubsection{\thesubsection.\arabic{subsubsection}}

\renewcommand\thesectiondis{\arabic{section}}
\renewcommand\thesubsectiondis{\thesectiondis.\arabic{subsection}}
\renewcommand\thesubsubsectiondis{\thesubsectiondis.\arabic{subsubsection}}


\hyphenation{op-tical net-works semi-conduc-tor}
\def\inputGnumericTable{}                                 %%

\lstset{
%language=C,
frame=single, 
breaklines=true,
columns=fullflexible
}
\begin{document}


\newtheorem{theorem}{Theorem}[section]
\newtheorem{problem}{Problem}
\newtheorem{proposition}{Proposition}[section]
\newtheorem{lemma}{Lemma}[section]
\newtheorem{corollary}[theorem]{Corollary}
\newtheorem{example}{Example}[section]
\newtheorem{definition}[problem]{Definition}

\newcommand{\BEQA}{\begin{eqnarray}}
\newcommand{\EEQA}{\end{eqnarray}}
\newcommand{\define}{\stackrel{\triangle}{=}}
\bibliographystyle{IEEEtran}
\raggedbottom
\setlength{\parindent}{0pt}
\providecommand{\mbf}{\mathbf}
\providecommand{\pr}[1]{\ensuremath{\Pr\left(#1\right)}}
\providecommand{\qfunc}[1]{\ensuremath{Q\left(#1\right)}}
\providecommand{\sbrak}[1]{\ensuremath{{}\left[#1\right]}}
\providecommand{\lsbrak}[1]{\ensuremath{{}\left[#1\right.}}
\providecommand{\rsbrak}[1]{\ensuremath{{}\left.#1\right]}}
\providecommand{\brak}[1]{\ensuremath{\left(#1\right)}}
\providecommand{\lbrak}[1]{\ensuremath{\left(#1\right.}}
\providecommand{\rbrak}[1]{\ensuremath{\left.#1\right)}}
\providecommand{\cbrak}[1]{\ensuremath{\left\{#1\right\}}}
\providecommand{\lcbrak}[1]{\ensuremath{\left\{#1\right.}}
\providecommand{\rcbrak}[1]{\ensuremath{\left.#1\right\}}}
\theoremstyle{remark}
\newtheorem{rem}{Remark}
\newcommand{\sgn}{\mathop{\mathrm{sgn}}}
\providecommand{\abs}[1]{\(\left\vert#1\right\vert\)}
\providecommand{\res}[1]{\Res\displaylimits_{#1}} 
\providecommand{\norm}[1]{\(\left\lVert#1\right\rVert\)}
%\providecommand{\norm}[1]{\lVert#1\rVert}
\providecommand{\mtx}[1]{\mathbf{#1}}
\providecommand{\mean}[1]{E\(\left[ #1 \right]\)}
\providecommand{\fourier}{\overset{\mathcal{F}}{ \rightleftharpoons}}
%\providecommand{\hilbert}{\overset{\mathcal{H}}{ \rightleftharpoons}}
\providecommand{\system}{\overset{\mathcal{H}}{ \longleftrightarrow}}
	%\newcommand{\solution}[2]{\textbf{Solution:}{#1}}
\newcommand{\solution}{\noindent \textbf{Solution: }}
\newcommand{\cosec}{\,\text{cosec}\,}
\providecommand{\dec}[2]{\ensuremath{\overset{#1}{\underset{#2}{\gtrless}}}}
\newcommand{\myvec}[1]{\ensuremath{\begin{pmatrix}#1\end{pmatrix}}}
\newcommand{\mydet}[1]{\ensuremath{}}
\numberwithin{equation}{subsection}
\makeatletter
\@addtoreset{figure}{problem}
\makeatother
\let\StandardTheFigure\thefigure
\let\vec\mathbf
\renewcommand{\thefigure}{\theproblem}
\def\putbox#1#2#3{\makebox[0in][l]{\makebox[#1][l]{}\raisebox{\baselineskip}[0in][0in]{\raisebox{#2}[0in][0in]{#3}}}}
     \def\rightbox#1{\makebox[0in][r]{#1}}
     \def\centbox#1{\makebox[0in]{#1}}
     \def\topbox#1{\raisebox{-\baselineskip}[0in][0in]{#1}}
     \def\midbox#1{\raisebox{-0.5\baselineskip}[0in][0in]{#1}}
\vspace{3cm}
\title{AI1103 - Assignment 2}
\author{G Vojeswitha - AI20BTECH11024}
\maketitle
\newpage
\bigskip
\renewcommand{\thefigure}{\theenumi}
\renewcommand{\thetable}{\theenumi}
Download all python codes from 
\begin{lstlisting}
https://github.com/Vojeswitha05/Probability_AI1103/blob/main/Assignment_2/simulation_2.py
\end{lstlisting}
%
and latex-tikz codes from 
%
\begin{lstlisting}
https://github.com/Vojeswitha05/Probability_AI1103/blob/main/Assignment_2/Latex_2.tex
\end{lstlisting}
\section{Problem 5.28}
Let X denote the number of hours you study during a randomly selected school day. The probability that X can take the values x, has the following form, where k is some unknown constant.
\begin{align}
\Pr\brak{X = x} = \left\{\begin{array}{lr}
0.1, &\text{if} x = 0\\
kx, & \text{if } x = 1 or 2\\
k(5-x), & \text{if } x = 3 or 4\\
0, & \text{otherwise}
\end{array}\right\}
\end{align}
\begin{enumerate}[label={\alph*)}]
    \item  Find the value of k.
    \item  What is the probability that you study at least two hours ? Exactly two hours? At-most two hours?
\end{enumerate}
\section{Solution}
 Expanding the given form , we get :\\
 Probability of studying x number of hours is as follows when x varies from 0 to 4, and is 0 for all other values of x.\vspace{3mm}
 \begin{center}
  
  \begin{table}[ht]
  
  \centering
  \begin{tabular}{|c|c|c|c|c|c|}
    \hline
    x &  0 & 1 & 2 & 3 & 4\\
    \hline
    $\Pr\brak{X=x}$ & 0.1& 1k & 2k & 2k & k\\
    \hline
    
\end{tabular} 
\caption{Probabilities in terms of k }
\end{table}
\end{center}
We know by definition,
\begin{align}
    \sum_{x = 0}^4 \Pr\brak{X = x} = 1 \label{eq 2.0.1}
\end{align}
By substituting the probabilities in \eqref{eq 2.0.1} we get,
\begin{align}
& \implies 0.1 + k + 2k + 2k + k = 1 \\
& \implies 6k = 0.9 \label{eq 2.0.3}
\end{align}
Therefore, from \eqref{eq 2.0.3}
\begin{align}
    k = 0.15
\end{align}
  Therefore the probability for x hours of study where \(0 \leq x \leq 4\), is as follows :
  
  \begin{table}[ht]
  
  \centering
  \begin{tabular}{|c|c|c|c|c|c|}
    \hline
    x &  0 & 1 & 2 & 3 & 4\\
    \hline
    $\Pr\brak{X=x}$ & 0.1& 0.15& 0.3 & 0.3 & 0.15\\
    \hline
    
\end{tabular} 
\caption{Probabilities after finding k}
\end{table}

\begin{figure}[ht]
    \centering
    \includegraphics[width=8cm]{PMF.png}
    \caption{Probability Mass Function (PMF)}
    \label{Figure_1}
\end{figure}

We know that, Cumulative Distributive Function (CDF) 
\begin{align}
    F(x) = \Pr\brak{X \le x}
\end{align}
\begin{table}[ht]
  
  \centering
  \begin{tabular}{|c|c|c|c|c|c|}
    \hline
    x &  0 & 1 & 2 & 3 & 4\\
    \hline
    $F(X)$ & 0.1& 0.25& 0.55 & 0.85 & 1\\
    \hline
    
\end{tabular} 
\caption{CDF}
\label{Table_2}
\end{table}

\begin{figure}[ht]
    \centering
    \includegraphics[width=8cm]{CDF.png}
    \caption{Cumulative Distributive Function (CDF) }
    \label{Figure_2}
\end{figure}

\vspace{3mm}
    \begin{enumerate}
        \item Probability of studying at least two hours 
        \begin{align}
            & = \sum_{k = 2}^4 \Pr\brak{X = k}\\
            & = \Pr\brak{1 < X \le 4}\\ 
            & =  F(4) - F(1)\label{eq 2.0.9}
        \end{align}
        Substituting CDF values of \eqref{Table_2} in \eqref{eq 2.0.9}
        \begin{align}
            &  = 1 - 0.25\\
            & = 0.75
        \end{align}
        
        \item Probability of studying exact two hours
        \begin{align}
            & = \Pr\brak{X = 2}\\
            & = 0.3
        \end{align}
        
        \item Probability of studying at most two hours 
        \begin{align}
          & = \sum_{x = 0}^2  \Pr\brak{X = x} = \Pr \brak{X \le 2}\\
           & = F(2)\label{eq 2.0.16}
        \end{align}
        By substituting probabilities in \eqref{eq 2.0.16}
        \begin{align}
            & = 0.55
        \end{align}
    \end{enumerate}
  Final solution :  
  
    \begin{center}
  
  \begin{tabular}{|c|c|c|}
    \hline
    $\Pr\brak{X \geq 2}$ &  $\Pr \brak{X = 2}$ & $\Pr\brak{X \leq 2}$\\
    \hline
     0.75& 0.3& 0.55 \\
    \hline
    Case1 &Case2 &Case3\\
    \hline
\end{tabular}
\end{center}
\begin{figure}[ht]
    \centering
    \includegraphics[width=8cm]{Theoritical_simulated.png}
    \caption{Comparison of theoretical and simulation values}
    \label{Figure_1}
\end{figure}
\end{document}
